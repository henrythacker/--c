\subsection{Test 1 - typecheck1.c}
\subsubsection{Input File}
\begin{lstlisting}[showstringspaces=false,breaklines=true,backgroundcolor=\color{light-gray}, captionpos=b]
/*Result: FAIL*/

/*
*	Sample --C Type checking #1
*	This example should FAIL
*
*	Henry Thacker
*/

int myfn(int a) {
	return a + 2;
}


int main() {
	return myfn(myfn);
}
\end{lstlisting}
\subsubsection{Purpose of Test}
This example is supposed to cause a failure in the interpreter. Unfortunately the TAC generator is likely to still generate a TAC representation for this input, because it is unable to type check parameters. As a result, the code generator will also go on to generate an invalid MIPS assembly program.
\subsubsection{Expected Result}
The expected output for this test $=$ Typecheck failure
\subsubsection{Interpreter Result}
The result from the interpreter is: Could not be interpreted
\subsubsection{Generated TAC}
\begin{lstlisting}[showstringspaces=false,breaklines=true,backgroundcolor=\color{light-gray}, captionpos=b]
BeginFn myfn
_myfn:
InitFrame 2
PopParam a
FnBody
_t1 = a + 2
Return _t1
EndFn
BeginFn main
_main:
InitFrame 1
FnBody
PrepareToCall 1
PushParam myfn
_t2 = CallFn _myfn
Return _t2
EndFn

\end{lstlisting}\subsubsection{Generated MIPS Assembly}
\begin{lstlisting}[showstringspaces=false,breaklines=true,backgroundcolor=\color{light-gray}, captionpos=b]
         
# Sun Jan 10 19:16:24 2010

.data
	EOL:	.asciiz "\n"
.text

_myfn:
	move $s7, $ra	# Store Return address in $s7
	li $a0, 24	# Store the frame size required for this AR
	jal mk_ar
	move $s0, $v0	# Store heap start address in $s0
	lw $a0, ($sp)	# Pop the parameter
	sw $a0, -4($fp)	# Write param into heap
	sw $s7, ($sp)	# Save return address in stack
	lw $t1, -4($fp)	# Load local variable
	addi $t0, $t1, 2
	move $v0, $t0	# Assign values
	sw $t0, -8($fp)	# Write out used local variable
	lw $ra, ($sp)	# Get return address
	add $sp, $sp, 4	# Pop return address from stack
	lw $fp, 4($s0)	# Load previous frame ptr
	lw $s0, 8($s0)	# Load dynamic link
	jr $ra	# Jump to $ra
_main:
	move $s7, $ra	# Store Return address in $s7
	li $a0, 20	# Store the frame size required for this AR
	jal mk_ar
	move $s0, $v0	# Store heap start address in $s0
	sub $sp, $sp, 4
	sw $s7, ($sp)	# Save return address in stack
	la $v0, _myfn	# Store address of function
	move $v1, $s0	# Store static link to call with
	jal rfunc	# Register fn variable
	move $t0, $v0	# Return fn descriptor address
	sub $sp, $sp, 4	# Move stack pointer
	sw $t0, ($sp)	# Write param into stack
	lw $v0, ($s0)	# Point callee to same static link as mine (caller)
	move $a1, $s0	# Pass dynamic link
	jal _myfn
	move $t0, $v0
	sw $t0, -4($fp)	# Write out used local variable
	lw $ra, ($sp)	# Get return address
	add $sp, $sp, 4	# Pop return address from stack
	lw $fp, 4($s0)	# Load previous frame ptr
	lw $s0, 8($s0)	# Load dynamic link
	jr $ra	# Jump to $ra
# Make a new activation record
# Precondition: $a0 contains total required heap size, $a1 contains dynamic link, $v0 contains static link
# Returns: start of heap address in $v0, heap contains reference to static link and old $fp value
mk_ar:
	move $s1, $v0	# Backup static link in $s1
	li $v0, 9	# Allocate space systemcode
	syscall	# Allocate space on heap
	move $s2, $fp	# Backup old $fp in $s2
	add $fp, $v0, $a0	# $fp = heap start address + heap size
	sw $s1, ($v0)	# Save static link
	sw $s2, 4($v0)	# Save old $fp
	sw $a1, 8($v0)	# Save dynamic link
	sw $a0, 12($v0)	# Save framesize
	jr $ra
# Make a new function variable entry
# Precondition: $v0 contains function address, $v1 contains static link
# Returns: address of allocated fn entry descriptor in $v0
rfunc:
	move $s1, $v0	# Backup fn address in $s1
	li $a0, 8	# Space required for descriptor
	li $v0, 9	# Allocate space systemcode
	syscall	# Allocate space on heap
	sw $s1, ($v0)	# Store fn address
	sw $v1, 4($v0)	# Store static link
	jr $ra
	.globl main
main:
	move $a1, $zero	# Zero dynamic link
	move $v0, $zero	# Zero static link
	jal _main
	move $a0, $v0	# Retrieve the return value of the main function
	li $v0, 1	# Print integer
	syscall
	li $v0, 4	# Print string
	la $a0, EOL	# Printing EOL character
	syscall
	li $v0, 10	# System exit
	syscall

\end{lstlisting}\subsubsection{SPIM Output}
\begin{verbatim}
SPIM Version 7.4 of January 1, 2009
Copyright 1990-2004 by James R. Larus (larus@cs.wisc.edu).
All Rights Reserved.
See the file README for a full copyright notice.
Loaded: /opt/local/share/spim/exceptions.s
268566550
\end{verbatim}
\subsubsection{Test Analysis}
\begin{description}
	\item[Interpreter] The interpreter gave the expected result - An integer parameter was expected to be passed to \verb!myfn!, so the interpreter raised an error.
	\item[TAC] Because the TAC generator does not type-check pushed parameters, it continues to generate a TAC translation for the invalid program.
	\item[MIPS] The TAC representation given to the code generator is incorrect, so the generated MIPS program returns an undefined response when run in SPIM.
\end{description}
