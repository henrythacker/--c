\subsection{Test 4 - fibonacci.c}
\subsubsection{Input File}
\begin{lstlisting}[showstringspaces=false,breaklines=true,backgroundcolor=\color{light-gray}, captionpos=b]
/*Result: 34*/

/*
*	Sample --C Fibonacci Programme
*	Henry Thacker
*/

int fib(int n) {
	if (n < 2) 
		return n;
	else
		return fib(n - 1) + fib(n - 2);
}

/* Main entry point */
int main() {
	return fib(9);
}
\end{lstlisting}\subsubsection{Purpose of Test}

\subsubsection{Expected Result}
The expected output for this test $=$ Result: 34
\subsubsection{Interpreter Result}
The result from the interpreter is: 34
\subsubsection{Generated TAC}
\begin{lstlisting}[showstringspaces=false,breaklines=true,backgroundcolor=\color{light-gray}, captionpos=b]
BeginFn fib
_fib:
InitFrame 7
PopParam n
FnBody
_t1 = n < 2
If _t1 Goto __if1true
_t3 = n - 1
PrepareToCall 1
PushParam _t3
_t4 = CallFn _fib
_t5 = n - 2
PrepareToCall 1
PushParam _t5
_t6 = CallFn _fib
_t2 = _t4 + _t6
Return _t2
Goto __if1end
__if1true:
Return n
__if1end:
EndFn
BeginFn main
_main:
InitFrame 1
FnBody
PrepareToCall 1
PushParam 9
_t7 = CallFn _fib
Return _t7
EndFn

\end{lstlisting}\subsubsection{Generated MIPS Assembly}
\begin{lstlisting}[showstringspaces=false,breaklines=true,backgroundcolor=\color{light-gray}, captionpos=b]
         
# Sun Jan 10 14:14:55 2010

.data
	EOL:	.asciiz "\n"
.text

_fib:
	move $s7, $ra	# Store Return address in $s7
	li $a0, 44	# Store the frame size required for this AR
	jal mk_ar
	move $s0, $v0	# Store heap start address in $s0
	lw $a0, ($sp)	# Pop the parameter
	sw $a0, -4($fp)	# Write param into heap
	sw $s7, ($sp)	# Save return address in stack
	lw $t1, -4($fp)	# Load local variable
	slti $t0, $t1, 2	# $c = $a < b
	bne $t0, $zero, __if1true
	sub $t2, $t1, 1
	sub $sp, $sp, 4	# Move stack pointer
	sw $t2, ($sp)	# Write param into stack
	sw $t0, -8($fp)	# Write out used local variable
	sw $t2, -16($fp)	# Write out used local variable
	lw $v0, ($s0)	# Point callee to same static link as mine (caller)
	move $a1, $s0	# Pass dynamic link
	jal _fib
	move $t0, $v0
	lw $t2, -4($fp)	# Load local variable
	sub $t1, $t2, 2
	sub $sp, $sp, 4	# Move stack pointer
	sw $t1, ($sp)	# Write param into stack
	sw $t0, -20($fp)	# Write out used local variable
	sw $t1, -24($fp)	# Write out used local variable
	lw $v0, ($s0)	# Point callee to same static link as mine (caller)
	move $a1, $s0	# Pass dynamic link
	jal _fib
	move $t0, $v0
	lw $t2, -20($fp)	# Load local variable
	add $t1, $t2, $t0
	move $v0, $t1	# Assign values
	sw $t0, -28($fp)	# Write out used local variable
	sw $t1, -12($fp)	# Write out used local variable
	lw $ra, ($sp)	# Get return address
	add $sp, $sp, 4	# Pop return address from stack
	lw $fp, 4($s0)	# Load previous frame ptr
	lw $s0, 8($s0)	# Load dynamic link
	jr $ra	# Jump to $ra
	j __if1end
__if1true:
	lw $t0, -4($fp)	# Load local variable
	move $v0, $t0	# Assign values
	lw $ra, ($sp)	# Get return address
	add $sp, $sp, 4	# Pop return address from stack
	lw $fp, 4($s0)	# Load previous frame ptr
	lw $s0, 8($s0)	# Load dynamic link
	jr $ra	# Jump to $ra
__if1end:
	lw $ra, ($sp)	# Get return address
	add $sp, $sp, 4	# Pop return address from stack
	move $v0, $zero	# Null return value
	lw $fp, 4($s0)	# Load previous frame ptr
	lw $s0, 8($s0)	# Load dynamic link
	jr $ra	# Jump to $ra
_main:
	move $s7, $ra	# Store Return address in $s7
	li $a0, 20	# Store the frame size required for this AR
	jal mk_ar
	move $s0, $v0	# Store heap start address in $s0
	sub $sp, $sp, 4
	sw $s7, ($sp)	# Save return address in stack
	li $a0, 9
	sub $sp, $sp, 4	# Move stack pointer
	sw $a0, ($sp)	# Write param into stack
	lw $v0, ($s0)	# Point callee to same static link as mine (caller)
	move $a1, $s0	# Pass dynamic link
	jal _fib
	move $t0, $v0
	sw $t0, -4($fp)	# Write out used local variable
	lw $ra, ($sp)	# Get return address
	add $sp, $sp, 4	# Pop return address from stack
	lw $fp, 4($s0)	# Load previous frame ptr
	lw $s0, 8($s0)	# Load dynamic link
	jr $ra	# Jump to $ra
# Make a new activation record
# Precondition: $a0 contains total required heap size, $a1 contains dynamic link, $v0 contains static link
# Returns: start of heap address in $v0, heap contains reference to static link and old $fp value
mk_ar:
	move $s1, $v0	# Backup static link in $s1
	li $v0, 9	# Allocate space systemcode
	syscall	# Allocate space on heap
	move $s2, $fp	# Backup old $fp in $s2
	add $fp, $v0, $a0	# $fp = heap start address + heap size
	sw $s1, ($v0)	# Save static link
	sw $s2, 4($v0)	# Save old $fp
	sw $a1, 8($v0)	# Save dynamic link
	sw $a0, 12($v0)	# Save framesize
	jr $ra
	.globl main
main:
	move $a1, $zero	# Zero dynamic link
	move $v0, $zero	# Zero static link
	jal _main
	move $a0, $v0	# Retrieve the return value of the main function
	li $v0, 1	# Print integer
	syscall
	li $v0, 4	# Print string
	la $a0, EOL	# Printing EOL character
	syscall
	li $v0, 10	# System exit
	syscall

\end{lstlisting}\subsubsection{SPIM Output}
\begin{verbatim}
SPIM Version 7.4 of January 1, 2009
Copyright 1990-2004 by James R. Larus (larus@cs.wisc.edu).
All Rights Reserved.
See the file README for a full copyright notice.
Loaded: /opt/local/share/spim/exceptions.s
34
\end{verbatim}\subsubsection{Test Analysis}

