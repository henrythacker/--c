\section{MIPS Assembler Compiler}
\label{sec:MIPS}
\subsection{Introduction}
Before embarking upon this project, very little was known about the MIPS architecture and code generation, in general. These were the main areas that were researched. The code generator that was produced takes in a list of Three Address Code instructions (as described fully in section \ref{sec:tacintro}) and outputs MIPS code that is designed to be run in the SPIM emulator\footnote{\url{http://pages.cs.wisc.edu/~larus/spim.html}}. MIPS code can be generated for most input programs, but for a few exceptions that will be mentioned at the end of this section. The main body of the code generator is another recursive procedure which walks the list of TAC instructions, generating code on the fly.

\subsection{General Ideas}
