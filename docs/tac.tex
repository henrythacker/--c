\chapter{Three Address Code}
\section{Introduction}
\label{sec:tacintro}
Throughout the design of the three address code, it was important to keep the purpose of intermediate representations in mind:

\begin{quotation}
	\ \\The \emph{intermediate code generation module} translates language-specific constructs in the AST into more general constructs \ldots One criterion for the level of the intermediate code is that it should be reasonably straightforward to generate machine code from it for various machines \ldots \cite{grune2000}.
\end{quotation}
\ \\
In order to genericise the instructions as much as possible, the instruction set provides no special instructions for \verb!WHILE! loops, inner functions or variable type specific operations. The instruction set is described in table \ref{tab:instruction-set}. The \textbf{overall} syntax of the TAC instructions will also be discussed briefly afterwards.
\ \\ \ \\
As mentioned in section \ref{sec:tacshortintro}, it is not currently possible to input TAC directly to the code generator. Table \ref{tab:instruction-set} only describes the syntax and semantics of the TAC instructions as generated from the AST provided by the parser. If TAC input \emph{was} entered directly by the user, the scheme would need to be consistent with the details provided below.
\ \\
\section{Instruction Set}
\begin{longtable}{|p{5.5cm}|p{3.5cm}|p{7cm}|}
	\caption[]{Instruction Set \label{tab:instruction-set}}\\
	\hline	
	\textbf{Instruction} & \textbf{TAC Type} & \textbf{Description} \\ \hline
	\endfirsthead
	\caption[]{(Continued from previous page)}\\	
	\hline \textbf{Instruction} & \textbf{TAC Type} & \textbf{Description} \\ \hline
	\endhead
	\verb!Label:! & TT\_LABEL & Labels can be made up of any alphanumeric characters. The AST $\rightarrow$ TAC translation utilises double underscore to denote jumps within \verb!IF! statements, e.g. \verb!__if1end! \\ \hline
	\verb!_FunctionLabel:! & TT\_FN\_DEF & Function labels are prefixed with a single underscore \\ \hline
	\verb![result = ] CallFn <FnName>! & TT\_FN\_CALL & \verb!<FnName>! is the function that should be invoked. The \verb![result = ]! portion is optional if no return value is expected \\ \hline
	\verb!InitFrame <FrameSize>! & TT\_INIT\_FRAME & \verb!<FrameSize>! is the \textbf{number} of locals in the local environment of this function. This instruction is important as it gives the code generator information about how much space to allocate for local variables \\ \hline
	\verb!PopParam <Var>! & TT\_POP\_PARAM & Each formal parameter must be popped as part of the function definition \\ \hline
	\verb!PushParam <Value>! & TT\_PUSH\_PARAM & Each actual parameter must be pushed as part of a function call. Parameters must be pushed in \textbf{reverse} order. If a function has parameters \verb!int a, int b!, the caller must push the value for \verb!b! before the value for parameter \verb!a!\\ \hline
	\verb!If <Cond.> Goto <TrueJmp>! & TT\_IF & \verb!<Cond.>! is the branch condition, this should be an integer, temporary or variable \verb!<TrueJmp>! is a label that will be jumped to when \verb!<Cond.>! is true. Enforcing a simple IF statement of this format (with zero denoting FALSE) means we do not have to have a multitude of different comparison functions. The comparison complexity is moved into the \verb!TT_OP! operation\\ \hline
	\verb!<Result> = <Value>! & TT\_ASSIGN & \verb!<Result>! is the name of a variable into which the value of \verb!<Value>! will be stored\\ \hline
	\verb!Goto <Label>! & TT\_GOTO & \verb!<Label>! is a label that will immediately be jumped to\\ \hline
	\verb!<Result> = <Op1> <Op> <Op2>! & TT\_OP & \verb!<Result>! is where the value of of the operation will be stored (temporary or variable), \verb!<Op1>! is the first operand, \verb!<Op2>! is the second operand and \verb!<Op>! is an operator: \verb`+, -, *, /, %, ==, >, <, <=, >=, !=`\\ \hline
	\verb!Return <Value>! & TT\_RETURN & \verb!<Value>! is the return value. A value may be omitted if the function has a \verb!VOID! return type\\ \hline
	\verb!PrepareToCall <ParamCount>! & TT\_PREPARE & \verb!<ParamCount>! is the number of parameters that are about to be pushed as part of a function application\\ \hline
	\verb!BeginFn <FnName>! & TT\_BEGIN\_FN & \verb!<FnName>! is the name of a function that is about to be defined\\ \hline	
	\verb!FnBody! & TT\_FN\_BODY & This statement is used after all of the formal parameters have been popped and denotes the start of the function body\\ \hline
	\verb!EndFn! & TT\_END\_FN & This instruction denotes the end of a function\\ \hline	
\end{longtable}

\section{Overall Syntax}

It is not enough to only consider the semantics and syntax of each individual instruction, but also to examine how these instructions can be combined. Each syntax rule is given in terms of a pseudo grammar-like language and a textual description. Although the grammar definitions may seem unwieldy on first glance, they help formally define my textual explanation and could easily be adapted to actually implement the TAC parser that is currently missing. The grammar rules presented here are \textbf{not} used as part of the project and simply serve to supplement the textual description.

\subsection{Function Declaration}
\label{sec:funcdec}
A function declaration consists of a \verb!BeginFn FnName! instruction, followed by the function label prefixed with an underscore: \verb!_FnName!. An \verb!InitFrame! instruction is then required (with the number of the function locals): \verb!InitFrame 10!. All parameters must now be ``popped" in separate \verb!PopParam! instructions. If the function takes no parameters then no \verb!PopParam! instructions are required. To indicate the start of the function body, the \verb!FnBody! instruction must follow. At this point, the TAC for the function body is appended. At the end of the body, a \verb!EndFn! instruction must be used to indicate the end of the function.

\begin{verbatim}
	EOL :=
	    : "\n"
	    | "\r" "\n"
	    | EOL "\r"
	    | EOL "\r" "\n"
	
	Pop_Param_List :=
	    :  TT_POP_PARAM EOL
	    |  Pop_Param_List TT_POP_PARAM EOL
	
	Function_Declaration :=
	    : TT_BEGIN_FN EOL	TT_FN_DEF EOL TT_INIT_FRAME EOL Pop_Param_List
	      TT_FN_BODY EOL TAC_Instructions TT_END_FN
	    | TT_BEGIN_FN EOL	TT_FN_DEF EOL TT_INIT_FRAME EOL TT_FN_BODY EOL 
	      TAC_Instructions TT_END_FN
\end{verbatim}

\subsection{Function Application}
A valid function application must first start with a \verb!PrepareToCall [ArgNumber]! instruction. Each of the actual parameter values are now ``pushed", in \textbf{reverse} order. Argument pushing is achieved through use of the \verb!PushParam! instruction. The number of required push operations is indicated by the value of \verb![ArgNumber]! in the \verb!PrepareToCall! instruction. If \verb![ArgNumber]! $ = 0$, then no \verb!PushParam! instructions are required. Finally, \verb!CallFn [FnName]! is used to invoke the function with the correct parameters.

\begin{verbatim}
	EOL :=
	    : "\n"
	    | "\r" "\n"
	    | EOL "\r"
	    | EOL "\r" "\n"
	
	Push_Param_List :=
	    :  TT_PUSH_PARAM EOL
	    |  Push_Param_List TT_PUSH_PARAM EOL
	
	Function_Application :=
	    : TT_PREPARE EOL Push_Param_List TT_FN_CALL EOL
	    | TT_PREPARE EOL TT_FN_CALL EOL
\end{verbatim}

\subsection{TAC Syntax}
With the two definitions above, it is possible to write rules which describe a valid TAC input. The grammar below a goal \verb!Valid_TAC! which we are trying to aim for when writing a valid programme in TAC representation. To meet the criteria for \verb!Valid_TAC!, there must be \textbf{at least} one function declaration. On either side of a function declaration, variable assignments and operations are permitted (i.e. the use of global variables). To examine what TAC is permissible within a function declaration, we must backtrack slightly. With the \textbf{Function Declaration} rule above (see section \ref{sec:funcdec}), we make a reference to \verb!TAC_Instructions! which is defined below. \verb!TAC_Instruction! is any instruction that is permitted within the body of a function. This rule allows us to nest function declarations, use \verb!IF! statements, jumps, label declarations and call other functions.
\ \\ \ \\
Although this language has now been completely specified, there are still some constructs which may not be backwards-compatible with the \mmc parser (please see figure \ref{fig:parsefail}). An equivalent version that will parse is presented in figure \ref{fig:parsefix}. Such statements (\verb!TT_OP!) \textbf{are} permissible in the TAC specification to allow for assigning a calculation to a global variable during variable initialisation (see the TAC equivalent in figure \ref{fig:parsefailtac}). This is an interesting aside, but not a problem as such, because we have met the target that all valid \mmc programs can be converted to a TAC representation with the instruction set that has been presented.
\begin{verbatim}
	EOL :=
	    : "\n"
	    | "\r" "\n"
	    | EOL "\r"
	    | EOL "\r" "\n"
	
	Variable_Declaration := 
	    : TT_ASSIGN EOL
	    | Variable_Declaration TT_ASSIGN EOL
	
	Variable_Operations :=
	    : TT_OP EOL
	    | Variable_Operations TT_OP EOL
		
	Variable_Use :=
	    : Variable_Declaration
	    | Variable_Operations
	    | Variable_Use Variable_Declaration
	    | Variable_Use Variable_Operations
	
	TAC_Instructions :=
	    : Variable_Use
	    | Function_Declaration
	    | Function_Application
	    | TT_IF EOL
	    | TT_GOTO EOL
	    | TT_RETURN EOL
	    | TT_LABEL EOL
	    | TAC_Instructions Variable_Use
	    | TAC_Instructions Function_Declaration
	    | TAC_Instructions Function_Application
	    | TAC_Instructions TT_IF EOL
	    | TAC_Instructions TT_GOTO EOL
	    | TAC_Instructions TT_RETURN EOL
	    | TAC_Instructions TT_LABEL EOL
	
	Valid_TAC :=
	    : Function_Declaration
	    | Variable_Use Function_Declaration
	    | Variable_Use Function_Declaration Variable_Use
	    | Valid_TAC Function_Declaration
	    | Valid_TAC Variable_Use

	Goal := Valid_TAC
	
\end{verbatim}

\begin{figure}[p]
	\begin{verbatim}
		int a = 15;
		a = a - 12; /* << --C parser will not allow this outside of a function body */
		            /* TAC syntax allows such statements outside + inside functions */
		int main() {
			return a;
		}
	\end{verbatim}
	\caption{Modifying global variables - \mmc}
	\label{fig:parsefail}
\end{figure}

\begin{figure}[p]
	\begin{verbatim}
		int a = 15 - 12;

		int main() {
			return a;
		}
	\end{verbatim}
	\caption{Fixed: Modifying global variables - \mmc}
	\label{fig:parsefix}
\end{figure}

\begin{figure}[p]
	\begin{verbatim}
		_t1 = 15 - 12 /* TT_OPs are allowed on global vars */
		a = _t1
		BeginFn main
		_main:
		InitFrame 0
		FnBody
		Return a
		EndFn
	\end{verbatim}
	\caption{Modifying global variables - TAC}
	\label{fig:parsefailtac}
\end{figure}

